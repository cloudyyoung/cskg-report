\section{Preliminary}
\label{sec:pre}


The purpose of the preliminary section is to formally define the problem and necessary background associated with the problem.  It is more common to have this section for theoretical papers, where all mathematical symbols and graph models are defined.  For system papers, this section is optional.  If your work needs intensive background in a certain area that would make either the ``Related Work'' or the ``Design'' section too long, you can consider adding the ``Preliminary'' section.

In LaTeX, all numbers and mathematical symbols should be enclosed by \$...\$ so that they stand out from the text.  For example, number formats like $34$ and $i$ stand out better than text formats like 34 and i in the LaTeX source.  Depending on the style file we used, they may appear different in the PDF file as well.  

When defining symbols, try to use a consistent font and pick meaningful symbols. Table \ref{pre-symbols} provides examples of good and bad conventions for defining symbols.  


When defining the symbols and the problem, be careful with the logic flow.  Always define symbols and terms before using them.  You may find special symbols in \LaTeX $ $ at \cite{latex-symbols, latex-symbols-pdf, dextify}.
