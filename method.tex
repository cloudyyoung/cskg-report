\section{Methodology}
\label{sec:method}

% We focus on a range of common code smells as identified by Fowler \etal\cite{Martin_1999}, and our approach is oriented towards developing a scalable and interpretable framework suitable for diverse and complex codebases.

% The methodology of this research is structured to address the challenge of detecting software code smells using Knowledge Graphs (KGs). It encompasses a series of steps, each designed to progressively develop and refine the tools and techniques necessary for effective code smell detection in large-scale software projects.

This section outlines the systematic approach employed in this research to address the challenge of detecting software code smells using Knowledge Graphs (KGs). The methodology encompasses several distinct but interrelated stages, each contributing towards the development of a comprehensive, efficient, and interpretable code smells detection framework. It consists of the following two stages:

\begin{description}
    \item[Stage I.] Since the use of knowledge graphs for code smells detection is novel application of knowledge graphs, the first step for the project is to identify a suitable knowledge base tool for implementing a knowledge graph for software code.
    \item[Stage II.] Develop on algorithm for finding out code smells using knowledge graphs. The data used to test this algorithm involves a selection of relative code smells from the \textit{Code Smells Catalog} by Jerzyk \etal\cite{Jerzyk_2023}, which essentially includes all smells from the textbook \textit{Refactoring: Improving the Design of Existing Code, Second Edition} (Chapter 3) by Fowler \etal\cite{Martin_2018}.
\end{description}


The selection of large-scale software projects is prioritized as the primary data source,  detailed in \autoref{tab:codebases}. These codebases are selected from GitHub, and due to their complex architectures and extensive use of diverse programming paradigms, making them ideal for studying the effectiveness of knowledge graph-based code smell detection. The selection criteria include projects with a substantial codebase  and active development histories, as detailed in \autoref{tab:codebases_criteria}. We ensure these projects are sourced from reputable, open-source repositories to maintain transparency and accessibility. This selection provides a comprehensive and challenging dataset, crucial for evaluating the robustness and scalability of our proposed knowledge graph approach in detecting code smells in real-world, large-scale software environments.

\begin{table}[h]
\centering
\begin{threeparttable}
\begin{tabular}{p{2cm}|l|l|l}
\hline
\textbf{Name} & \textbf{LOC} & \textbf{Accessed Date} & \textbf{SHA}  \\ \hline
\href{https://github.com/huggingface/transformers}{Transformers (Hugging Face)}     & 1.22M  &  November 17, 2023 & 638d49983      \\ \hline
\end{tabular}
\caption{Selected large-scale project codebases}
\label{tab:codebases}
\end{threeparttable}
\end{table}


\begin{table}[h]
\centering
\begin{threeparttable}
\begin{tabular}{lp{6.3cm}}
\toprule
\textbf{Criterion} & \textbf{Description} \\ \midrule
CR1        & The primary language is Python, with a ratio of over 90\% among all codebase languages. \\ \midrule
CR2        & The total lines of code (LOC) is at least 1 million, including empty lines and comments. \\ \midrule
CR3        & The project is reputable by having at least 50k stars on GitHub. \\ \midrule
CR4        & There are at least 10,000 nodes and 20,000 edges in the transformed knowledge graph of the codebase. \\ 

\bottomrule
% Add more criteria as needed
\end{tabular}

\begin{tablenotes}
\item \small Note: The selection criteria for the codebases were applied in accordance with the research time-frame.
\end{tablenotes}

\caption{Codebase selection criteria}
\label{tab:codebases_criteria}

\end{threeparttable}
\end{table}



% Overview Illustration
\begin {figure}%[!hbtp]
\centering

\resizebox{0.6\linewidth}{!}{
\begin{tikzpicture}[node distance=2cm]

\node (codebase) [component] {Codebase};

\node (ast) [intermediate, below of=codebase] {Abstract Syntax Tree \\ \& Inference};
\node (class) [intermediate, below of=ast, xshift=-2.2cm, yshift=0.4cm] {Class};
\node (function) [intermediate, below of=ast, xshift=0cm, yshift=0.4cm] {Function};
\node (more) [intermediate, below of=ast, xshift=2.2cm, yshift=0.4cm] {...};

\node (kg) [component, below of=function] {Knowledge Graph};

% Arrows
\path (codebase) edge[arrow] node[on arrow] {Parse} (ast);

\path (ast) edge[arrow] (class);
\path (ast) edge[arrow] (function);
\path (ast) edge[arrow] (more);


\path (class) edge[arrow] (kg);
\path (function) edge[arrow] (kg);
\path (more) edge[arrow] (kg);


\end{tikzpicture}
}
\caption{Transformation of codebases into KG}
\label{fig:stage-1-overview}
\end{figure}



\begin{table*}[h]
\centering
\begin{threeparttable}

\begin{tabular}{p{1.6cm}p{2.5cm}p{2cm}p{10cm}}
\toprule
\textbf{Type} & \textbf{Label} & \textbf{Property} & \textbf{Description} \\
\cmidrule{1-1}\cmidrule{2-2}\cmidrule{3-3}\cmidrule{4-4}
\multirow{18}{*}[-28pt]{Entity} & \multirow{3}{*}[-28pt]{Module} & name & A string of identifier. \\
\cmidrule{3-3}\cmidrule{4-4}
 &  & qualified\_name & Identifier's complete path including its scope. \\
\cmidrule{3-3}\cmidrule{4-4}
 &  & file\_path & Code location. \\
\cmidrule{2-2}\cmidrule{3-3}\cmidrule{4-4}
 & \multirow{4}{*}[-0.8cm]{Class} & name & - \\
\cmidrule{3-3}\cmidrule{4-4}
 &  & qualified\_name & - \\
\cmidrule{3-3}\cmidrule{4-4}
 &  & is\_abstract & Whether if is an abstract class. \\
\cmidrule{3-3}\cmidrule{4-4}
 &  & file\_path & - \\
\cmidrule{2-2}\cmidrule{3-3}\cmidrule{4-4}
 & \multirow{4}{*}[-0.8cm]{Function} & name & - \\
\cmidrule{3-3}\cmidrule{4-4}
 &  & qualified\_name & - \\
\cmidrule{3-3}\cmidrule{4-4}
 &  & type & One of function, method, classmethod or staticmethod. \\
\cmidrule{3-3}\cmidrule{4-4}
 &  & file\_path & - \\
\cmidrule{2-2}\cmidrule{3-3}\cmidrule{4-4}
 & \multirow{3}{*}[-0.8cm]{Method (Function)} & name & - \\
\cmidrule{3-3}\cmidrule{4-4}
 &  & qualified\_name & - \\
\cmidrule{3-3}\cmidrule{4-4}
 &  & file\_path & - \\
\cmidrule{2-2}\cmidrule{3-3}\cmidrule{4-4}
 & \multirow{4}{*}[-0.8cm]{Variable} & name & - \\
\cmidrule{3-3}\cmidrule{4-4}
 &  & qualified\_name & - \\
\cmidrule{3-3}\cmidrule{4-4}
 &  & access & One of public, protected or private. \\
\cmidrule{3-3}\cmidrule{4-4}
 &  & file\_path & - \\
\bottomrule
\end{tabular}

\begin{tabular}{p{1.6cm}p{1.7cm}p{2cm}p{4.3cm}p{6.5cm}}
\toprule
\textbf{Type} & \textbf{Name} & \textbf{Label} & \textbf{Relations} & \textbf{Attributes} \\
\cmidrule{1-1}\cmidrule{2-2}\cmidrule{3-3}\cmidrule{4-4}\cmidrule{5-5}
\multirow{14}{*}[-0.8cm]{Relationship} & Call & CALLS & Function USES Function &  \\
\cmidrule{2-2}\cmidrule{3-3}\cmidrule{4-4}\cmidrule{5-5}
 & Inheritance & INHERITS & Class INHERITS Class &  \\
\cmidrule{2-2}\cmidrule{3-3}\cmidrule{4-4}\cmidrule{5-5}
 & \multirow{6}{*}[-0.8cm]{Containment} & \multirow{6}{*}[-0.8cm]{CONTAINS} & Module CONTAINS Class &  \\
\cmidrule{4-4}\cmidrule{5-5}
 &  &  & Module CONTAINS Function &  \\
\cmidrule{4-4}\cmidrule{5-5}
 &  &  & Module CONTAINS Variable &  \\
\cmidrule{4-4}\cmidrule{5-5}
 &  &  & Class CONTAINS Method &  \\
\cmidrule{4-4}\cmidrule{5-5}
 &  &  & Class CONTAINS Variable &  \\
\cmidrule{4-4}\cmidrule{5-5}
 &  &  & Function CONTAINS Variable &  \\
\cmidrule{2-2}\cmidrule{3-3}\cmidrule{4-4}\cmidrule{5-5}
 & Parameter & TAKES & Function TAKES Class &  \\
\cmidrule{2-2}\cmidrule{3-3}\cmidrule{4-4}\cmidrule{5-5}
 & Return & RETURNS & Function RETURNS Class &  \\
\cmidrule{2-2}\cmidrule{3-3}\cmidrule{4-4}\cmidrule{5-5}
 & Instantiation & INSTANTIATES & Class INSTANTIATES Variable &  \\
\cmidrule{2-2}\cmidrule{3-3}\cmidrule{4-4}\cmidrule{5-5}
 & \multirow{3}{*}[-0.8cm]{Utilization} & \multirow{3}{*}[-0.8cm]{USES} & Module USES Variable &  \\
\cmidrule{4-4}\cmidrule{5-5}
 &  &  & Class USES Variable &  \\
\cmidrule{4-4}\cmidrule{5-5}
 &  &  & Function USES Variable &  \\
\bottomrule
\end{tabular}



% \begin{tablenotes}
\small
% \item Note: The selection criteria for the codebases were applied in accordance with the research timeframe.
% \end{tablenotes}
\caption{Scheme of knowledge graph}
\label{tab:ents-and-rels}
\end{threeparttable}
\end{table*}






% Section 3.1 
\subsection{Transformation into Knowledge Graphs}
This stage involves converting software codebases into Knowledge Graphs, a process crucial for capturing the essential elements of the software's structure and semantics. \autoref{fig:stage-1-overview} provides a visual representation of this stage.

\subsubsection{Abstraction of Codebase} 
To begin with, a critical first step involves extract a comprehensive abstract model of the existing codebase. This extraction is important for gaining a deeper understanding of how various elements within the codebase, such as classes and functions, are interconnected and interact with each other. 


% AST
The \textit{abstract syntax tree} (AST) is a fundamental internal data structure in programming that captures the core structure of a program. It serves as the initial basis for conducting semantic analysis of the program with its enriched information. The term "abstract" implies that the AST abstracts away specific parsing details \cite{Thain_2021}. For the purpose of constructing a KG, it is suitable to use AST for several reasons:  

\begin{description}
  \item[Structured Representation.] AST provides a tree-like structure built from constructs like functions and variables, which represents the hierarchical syntax of the programming language used in the codebase. This structure is inherently suitable for conversion into a knowledge graph.
  \item[Semantic.] AST abstracts away from the surface syntax (such as punctuation and keywords) and focuses on the semantics of the code, which allows for a clearer understanding of the code's behavior and logic.
  \item[Scalability.] ASTs can handle large and complex codebases efficiently, making them suitable for creating extensive knowledge graphs that can encapsulate vast amounts of information.
\end{description}


% Inferences
\textit{Type inference} refers to the process in a typed programming language where the compiler deduces the types of expressions and subexpressions without requiring the programmer to explicitly annotate every element with its type. This is achieved by strategically placing type information at critical points in the program, such as for local variables, function arguments, and function results. Given this limited type information and the known types of variables and basic constants, the compiler can infer the types of other expressions and statements within the program  \cite{Cardelli_1985}. 

This mechanism helps in figuring out the types in the code where explicit type declarations are not provided, as seen in dynamically typed languages like Python. Furthermore, this type inference helps in recognizing related nodes within the codebase, and enhances graph's accuracy and utility in representing the code's structure and relationships.



% By extracting these abstracted information, we can scrutinize the code structure and behavior more effectively, which helps in identifying patterns and anomalies that may indicate code smells. This process of abstraction is a precursor to more detailed analysis, setting the stage for effective identification and resolution of potential issues within the code.



\subsubsection{Composition of Knowledge Graph}
By employing AST and type inference, the codebase is converted into meaningful, interconnected data that is primed for integration into a KG. \autoref{tab:ents-and-rels} outlines all entities and relationships that ultimately featured in the KG, detailing how entities are connected through the relationships.

To address the issue of functions and classes having identical names within different scopes, the system not only uses a \texttt{name} property but also a \texttt{qualified\_name} property. The \texttt{qualified\_name} is akin to a full address, detailing where in the code hierarchy an entity resides. It includes the module name and namespace, \eg \texttt{request.cookies.RequestsCookieJar}, ensuring each node is uniquely identified regardless of similar names in other scopes in the codebase. 





% Section 3.2
\subsection{Detection with Knowledge Graph}
In this stage, methods for identifying code smells within the knowledge graph are implemented. By leveraging the KG's structured and enhanced depiction of the codebase, along with converting pattern-based definitions into Cypher queries, code smells are identified.


From the \textit{code smells catalog} (referred as \textit{catalog}) by Jerzyk \etal, a new taxonomy for code smells is introduced and categorizes code smells by three groupings: \textit{Obstruction}, \textit{Expanse}, and \textit{Occurrence} \cite{Jerzyk_2023}. 

The approach in this work with the knowledge graph is particularly adept at identifying code smells in the \textit{Expanse} group with \textit{\textbf{Between}} label. 

The \textit{Expanse} grouping is defined by the scope of the code smells. It determines whether these smells are confined \textit{within} a single class or whether their detection necessitates a broader perspective, implying that these smells extend \textit{between} classes \cite{Jerzyk_2023}.

The choice of focusing on the \textit{Between} code smells is strategic, because knowledge graphs are effective in illustrating the relationships between entities, thereby facilitating the identification of these inter-class code smells.


The \textit{Cypher} query language is a declarative graph query language that allows for efficient querying and updating of graph databases \cite{neo4j_cypher_overview}. This query language is used for representation of data interaction in the knowledge graph in the subsequent sections.


To formalize each code smell, definitions are consolidated from these primary sources: 
\begin{enumerate*}[label={\alph*)},font={\color{cyan!50!black}\bfseries}]
\item the general definition by Fowler \etal in their 2018 book \cite{Martin_2018}
\item the general definition by Martin in his 2008 book \cite{Martin_2008}
\item the \textit{catalog} definition by Jerzyk \etal \cite{Jerzyk_2023}
\item the pattern-based approach by Zhang \etal \cite{Zhang_2008}.
\end{enumerate*}
Building on these previous definitions, this work introduces its own tailored to knowledge graphs and the Python language, offering a context-specific interpretation for smells detection.


\subsection{Speculative Generality}

\begin{description}[align=left, labelwidth=2.4cm]
  \item [Obstruction] Dispensable
  \item [Occurence] Unnecessary Complexity
  \item [Expanse] Between Classes 
\end{description}

Developers often add extra features, adding complexity for potential future scenarios "just in case", but they may never materialize, making the code more difficult to understand and maintain without there being any real benefit \cite{Martin_2018}. This tendency is rooted in human psychology and, despite good intentions, it leads to cluttered code \cite{Jerzyk_2023}.

Zhang \etal reformulated the definition into a pattern-oriented approach, encompassing two situations \cite{Zhang_2008}. \autoref{def:speculative-generality-1} and \ref{def:speculative-generality-2} are elaborated to adopt the context in this work accordingly.

\begin{definition}
A class is an abstract class\footnote{Original definition includes references to both an abstract class and an interface. However, given that Python supports multiple inheritance and does not utilize interfaces, the term "interface" is excluded.}, and this class is not inherited or is only inherited by one class.
\label{def:speculative-generality-1}
\end{definition}

\begin{definition}
A class includes at least one method that has at least one parameter that is not used.
\label{def:speculative-generality-2}
\end{definition}

Based on the definitions provided, the subsequent Cypher queries are formulated in accordance:

\begin{minted}{cypher}
MATCH (a:Class {is_abstract: true})
OPTIONAL MATCH (a)<-[:INHERITS]-(inheriting_class)
WITH a, 
     COUNT(inheriting_class) AS inherits_count
WHERE inherits_count <= 1
RETURN a.name AS class_name, 
       inherits_count
\end{minted}


\begin{minted}{cypher}
MATCH (c:Class)-[:CONTAINS]->(m:Method)
               -[:TAKES]->(p:Variable)
WHERE NOT (m)-[:USES]->(p)
RETURN c.name AS class_name, 
       m.name AS method_name, 
       p.name AS unused_parameter_name
\end{minted}

With the queries, classes that are observed with the traits of this code smells are found out.

\subsection{Base Class depends on Subclass}

\begin{description}[align=left, labelwidth=2.4cm]
  \item [Obstruction] Object Oriented Abusers
  \item [Occurence] Interfaces
  \item [Expanse] Between Classes
\end{description}

This case is originally proposed by Martin in his 2008 book \cite{Martin_2008}. It describes a situation where a base class is designed in a way that it requires knowledge about its subclasses, leading to violation of the principle of modular and independent design. This implementation approach should be avoided because it ideally allows child classes to be deployable and maintainable independently of their parent class, promoting modularity. When a subclass undergoes changes, it should not force a change of its superclass, thereby reducing the maintenance effort and scope needed \cite{Jerzyk_2023}. This code smell is often associated with the \textit{Shotgun Surgery}, which occurs when a single change affects multiple parts of a system \cite{Jerzyk_2023}. It may be considered an early indicator of a code smell that, if left unaddressed, often evolves into the latter.


To adapt to a definition that aligns with the methodology of this work, it suggests that this code smell occurs when a parent class has dependency on its child class, as defined in \autoref{def:base-class}.

\begin{definition}
A relationship exists between a parent class and its child class, other than an inheritance from the parent to the child class.
\label{def:base-class}
\end{definition}


\begin{minted}{cypher}
MATCH (parent:Class)-[r]->(child:Class)
WHERE NOT type(r) = "INHERITS" 
      AND (parent)-[:INHERITS]->(child)
RETURN parent.name AS parent_class, 
       child.name AS child_class, 
       type(r) AS relationship_type
\end{minted}

\subsection{Data Clumps}

\begin{description}[align=left, labelwidth=2.4cm]
  \item [Obstruction] Bloaters
  \item [Occurence] Data
  \item [Expanse] Between Classes 
\end{description}


When groups of variables frequently appear together throughout a codebase, it would be more efficiently managed as a single object \cite{Martin_2018}. The concept suggests that data items consistently used together, yet not organized as a unified entity, should be encapsulated within a class to improve convenience and coherence. An example would be the coordinate \texttt{x} and \texttt{y} values held separately rather than in a \texttt{Coordinate} object.


Zhang \etal reinterpreted the smell's definition into a pattern-based definition, which is divided into two situations \cite{Zhang_2008}. Following this framework, definitions suitable for the context of this work are presented, as detailed in \autoref{def:data-clumps-1} and \ref{def:data-clumps-2}.

\begin{definition}
More than three data fields consistently appear together across multiple classes, and these fields share the same signatures, \ie identical names, data types and access\footnote{Original definition refers to "access modifiers", but Python does not strictly have access modifier. Instead, Python employs name mangling, \ie the underscore symbol, to indicate access control levels for class data members or member functions. Therefore, the term "access" is adopted.}, regardless of their arrangement.
\label{def:data-clumps-1}
\end{definition}


\begin{definition}
More than three input parameters consistently appear together in the declarations of multiple methods, and these parameters share identical signatures, \ie identical names and data types, regardless of their arrangement.
\label{def:data-clumps-2}
\end{definition}


\textit{Frequent itemset mining} is a data mining process aimed at finding recurrent patterns, or itemsets, in a dataset where the occurrence of these itemsets exceeds a user-specified minimum support threshold \cite{Agrawal_1996}. In this context, a "transaction" refers to a single record or instance in the dataset that contains a set of items \cite{Toivonen_2010}. For example, in a market basket analysis, a transaction would represent a customer's shopping basket, comprising various products purchased together during a single shopping trip. Each item in the basket is equivalent to an item in the set, and the collection of all items bought together forms the transaction. 

The \textit{Apriori algorithm}, introduced by Agrawal and Srikant in 1996 \cite{Agrawal_1996}, is a classic algorithm used for frequent itemset mining. It operates on the principle that all subsets of a frequent itemset must also be frequent, known as the \textit{Apriori property}. It iteratively expands candidate itemsets, starting from individual items and progressively combining them, while pruning itemsets below the minimum support threshold. This iterative expansion and pruning continue until no further frequent itemsets can be identified, efficiently narrowing the search space and reducing computational effort \cite{Toivonen_2010}. 

The concepts outlined in the above definitions can be addressed though \textit{frequent itemset mining}, particularly for identifying recurring collections of Variable entities within certain entities. The approach starts by aggregating all Variable sets, and then utilizes the \textit{Apriori Algorithm} to identify those sets that frequently occur:

\begin{minted}{cypher}
MATCH (c:Class)-[:CONTAINS]->(v:Variable)
RETURN c.name AS class_name, 
       COLLECT(v.name) AS variables

MATCH (f:Function)-[:TAKES]->(v:Variable)
RETURN f.name AS function_name, 
       COLLECT(v.name) AS variables
\end{minted}

Upon constructing the subgraph, the next phase involves utilizing the \textit{Apriori Algorithm} to analyze these structured sets of variables. Each class, along with its functions and variables, is treated as a unique transaction. The algorithm is then applied to identify frequent patterns of variable usage across these transactions. The minimum support threshold is crucial here; it determines the minimum frequency at which a particular set of variables must appear across all classes to be considered a frequent itemset.

In this method, it is possible to uncover not only the most commonly used variables but also how these variables are shared and utilized among different classes and functions, revealing potential areas for optimization and refactoring within the codebase.


\subsection{Insider Trading} 

\begin{description}[align=left, labelwidth=2.4cm]
  \item [Obstruction] Data Dealers
  \item [Occurence] Responsibility
  \item [Expanse] Between Classes 
\end{description}

% Classes and modules should minimize their knowledge about each other's internal workings. Fowler \etal criticized the excessive interdependency by stating that "classes spend too much time delving in each other's private parts" \cite{Martin_2018}. Originally termed \textit{Inappropriate Intimacy} in 1999 \cite{Martin_1999}, this code smell has been updated to \textit{Insider Trading} in 2018 \cite{Martin_2018}, reflecting a shift from class-specific concerns to a broader module context while maintaining the core idea. The essence of the critique remains that classes or modules should avoid excessively sharing implementation details and data, preserving encapsulation and modularity.

This smell refers to a code smell where modules or classes interchange too much information and implementation details, leading to excessive knowledge about each other's inner workings. Initially identified as \textit{Inappropriate Intimacy} in 1999 \cite{Martin_1999}, the term was updated in 2018 to reflect the generalization from classes to modules and emphasize the issue's nature \cite{Martin_2018}. This concept highlights the problem of software components delving too deeply into each other's private parts, violating principles of encapsulation and modularity \cite{Jerzyk_2023}.

...



% To describe this code smell in a pattern-based definition, this code smell is identified in situations where excessive data exchange occurs between two classes, typically evidenced by the classes passing instances of themselves to each other, indicating a high level of coupling. 

% % This phenomenon often arises from class inheritance, where subclasses may have access to more information from their parent classes than is desirable \cite{Martin_2018}. This definition encompasses such scenarios, highlighting the issue of overly intimate class relationships.

% \begin{definition}
% Methods in two classes each accept instances of the other class as parameters.
% \label{def:insider-trading}
% \end{definition}

% Which translates to the following Cypher query:

% \begin{minted}{cypher}
% MATCH (c1:Class)-[:CONTAINS]->(m1:Method)-[:TAKES]->(c2:Class),
%       (c2)-[:CONTAINS]->(m2:Method)-[:TAKES]->(c1)
% WHERE NOT c1 = c2
% RETURN c1, m1, c2, m2
% \end{minted}





% \subsubsection{Middle Man}
% \subsubsection{Tramp Data}
% \subsubsection{Refused Bequest}
% \subsubsection{Parallel Inheritance Hierarchies}
% \subsubsection{Base Class depends on Subclass} % Base class should not have relationships out to subclass..?
